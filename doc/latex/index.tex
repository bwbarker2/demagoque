\hypertarget{index_license}{}\section{License}\label{index_license}
Copyright (C) 2011 Brent W. Barker

This program is free software: you can redistribute it and/or modify it under the terms of the GNU General Public License as published by the Free Software Foundation, either version 3 of the License, or (at your option) any later version.

This program is distributed in the hope that it will be useful, but WITHOUT ANY WARRANTY; without even the implied warranty of MERCHANTABILITY or FITNESS FOR A PARTICULAR PURPOSE. See the GNU General Public License for more details.

You should have received a copy of the GNU General Public License along with this program (gpl-\/3.0.txt). If not, see $<$\href{http://www.gnu.org/licenses/}{\tt http://www.gnu.org/licenses/}$>$.\hypertarget{index_author}{}\section{Author}\label{index_author}
Brent W. Barker\par
 barker at nscl dot msu dot edu\par
 National Superconducting Cyclotron Laboratory\par
 Michigan State University\par
 1 Cyclotron, East Lansing, MI 48824-\/1321\hypertarget{index_coords}{}\section{Coordinate system}\label{index_coords}
In the rotated coordinate system, from the original x,x' system, the following coordinate system is used:

{\ttfamily  xa = (x+x')/2 \par
 xr = (x-\/x') \par
 ka = (k+k')/2 \par
 kr = (k-\/k') \par
 }

In the spatial coordinates, the density matrix is of the form {\ttfamily denmat(xa,xr)}. In the spectral coordinates, it is of the form {\ttfamily denmat(kr,ka)}. Note that the order of relative and absolute is switched. This is because the Fourier transform in these coordinates associates the xr coordinate with ka, and xa with kr.\hypertarget{index_potentials}{}\section{Potentials}\label{index_potentials}
These are selected with the potInitial and potFinal variables. The initial state should probably be an eigenstate of {\ttfamily potInitial}. The {\ttfamily potFinal} potential is that which is used for the time evolution. If potInitial and potFinal are different, then the system is switched adiabatically from the initial to the final potential. Here is the definition of the different potInitial/Final integers:

\begin{TabularC}{2}
\hline
code &definition  \\\cline{1-2}
-\/1 &no potential at all, free space  \\\cline{1-2}
0 &external harmonic oscillator centered at x=0 \\\cline{1-2}
1 &nonlocal meanfield harmonic oscillator \\\cline{1-2}
2 &Skyrme-\/like contact potential (local density dependent)  \\\cline{1-2}
3 &same as pot=0, but with exact evolution from Chin, Krotsheck, Phys Rev E72, 036705 (2005)  \\\cline{1-2}
\end{TabularC}
\hypertarget{index_options}{}\section{Options}\label{index_options}
Options are given at the end of the code. Eventually the potentials will be options as well. Each line consists of a space-\/separated list of parameters. The first is a string that defines the option to set, followed by a list of parameters for that option. Options can be safely commented out with a starting '!' \begin{TabularC}{2}
\hline
option &definition 

\\\cline{1-2}
initialSeparation &initial separation between center-\/of-\/masses of fragments in fm. Currently rounds displacement to nearest grid point, rather than interpolating.\par


Arguments: real$\ast$8 initialSeparation

\\\cline{1-2}
splitOperatorMethod &time evolve using SOM. Parameter is order of method. Available orders are 3 and 5. Formulae from A.D.Bandrauk, H. Shen, J. Chem. Phys. 99, 1185 (1993).\par


Arguments: integer splitOperatorMethod

\\\cline{1-2}
useFlipClone &create symmetric system by adding to the system its conjugate, reflected about the xa axis. Option 'initialSeparation' must be set.\par


Arguments: None

\\\cline{1-2}
useImCutoff &imaginary off-\/diagonal cutoff. \par


Arguments: real$\ast$8 cutoff\_\-w0, real$\ast$8 cutoff\_\-x0, real$\ast$8 cutoff\_\-d0  \\\cline{1-2}
\end{TabularC}
